\documentclass{article}
\usepackage[utf8]{inputenc}
\usepackage[T2A]{fontenc}
\usepackage{xcolor}
\title{SmartCalc Documentation}
\date{December 2023}
\author{charlsib}
\begin{document}
\maketitle


\section{Introduction to backend}

\subsection{\textbf{Файл parse\_utility\_functions.c}}

\begin{itemize}
    \item  @brief - Эта функция переводит алгебраическое выражение в инфиксной нотации - в постфиксную.

    \item  @param expression - алгебраическое выражение в инфиксной нотации

    \item  @return  - алгебраическое выражение в постфиксной нотациии
\end{itemize}

\textcolor{blue}{char * parse\_infix\_expr(char *expression);}


{\color{red} \rule{\linewidth}{0.5mm}}

\begin{itemize}
\item @brief Функция проверяет, является ли символ цифрой

\item @param ch - проверяемый символ

\item @return - 1 - True, 0 - False
\end{itemize}

\textcolor{blue}{int s21\_isdigit(char ch);}


{\color{red} \rule{\linewidth}{0.5mm}}

\begin{itemize}
\item @brief Функция добавляет символ в конец строки

\item @param str - строка, к которой будет добавлен символ

\item @param ch - символ
\end{itemize}

\textcolor{blue}{void add\_char(char *str, char ch);}


{\color{red} \rule{\linewidth}{0.5mm}}

\begin{itemize}
\item @brief Эта функция проверяет, является ли строка, с заданным смещением, началом какой-либо функции.

\item @param str - Проверяемая строка

\item @param pos - Смещение

\item @return - True/False
\end{itemize}

\textcolor{blue}{int isfunction(char *str, int pos);}


{\color{red} \rule{\linewidth}{0.5mm}}

\begin{itemize}
\item @brief Эта функция определяет математическую функцию с заданной позиции строки

\item @param str - проверяемая строка

\item @param pos - заданное смещение строки

\item @param offset - то смещение, на которое нужно сместить строку (дабы считать функцию asin 1 раз, а не два).

\item @return результат - определенная функция (определяется одним символом из массива FUNCTIONS\_SUBSTITUTIONS)
\end{itemize}

\textcolor{blue}{char determine\_function(char *str, int pos, int *offset);}


{\color{red} \rule{\linewidth}{0.5mm}}

\begin{itemize}
\item @brief Функция проверяет, является ли данный символом оператором

\item @param ch - проверяемый символ

\item @return - True/False
\end{itemize}

\textcolor{blue}{int isoperator(char ch);}


{\color{red} \rule{\linewidth}{0.5mm}}

\begin{itemize}
\item @brief Функция проверяет приоритет данного символа

\item @param ch проверямый символ

\item @return значение приоритета оператора.
\end{itemize}

\textcolor{blue}{int get\_priority(char ch);}


{\color{red} \rule{\linewidth}{0.5mm}}

\begin{itemize}
\item @brief Функция проверяет является ли проверяемый символ функцией

\item @param ch проверяемый символ

\item @return True/False
\end{itemize}

\textcolor{blue}{int isfunction\_stack(char ch);}


{\color{red} \rule{\linewidth}{0.5mm}}



\subsection{\textbf{stack\_calc.c}}
\begin{itemize}
\item @brief Эта функция пушит значение item на вершину стека

\item @param item 
\end{itemize}

\textcolor{blue}{void push\_calc(double item);}


{\color{red} \rule{\linewidth}{0.5mm}}

\begin{itemize}
\item @brief Эта функция вытаскивает значение с вершины стека (если оно существует)

\item @return значение на вершине стека (либо 0)
\end{itemize}

\textcolor{blue}{double pop\_calc();}


{\color{red} \rule{\linewidth}{0.5mm}}

\subsection{\textbf{stack.c}}

\begin{itemize}

\item @brief Функция пушит элемент item на вершину стека

\item @param item 
\end{itemize}

\textcolor{blue}{void push(char item);}


{\color{red} \rule{\linewidth}{0.5mm}}

\begin{itemize}
\item @brief Функция удаляет элемент, находящийся на вершине, если он есть и возвращает его

\item @return элемент с вершины (либо 0, если нет элемента)
\end{itemize}

\textcolor{blue}{char pop();}

{\color{red} \rule{\linewidth}{0.5mm}}

\subsection{\textbf{calculate\_rpn.c}}

\begin{itemize}
\item @brief Функция совершает операцию между двумя операндами и одним оператором

\item @param value\_1 первый операнд

\item @param value\_2 второй операнд

\item @param token оператор

\item @param result результат операции

\item @return код ошибки (0 - норм, 2 - не норм, деление на ноль)
\end{itemize}

\textcolor{blue}{int calculate\_binary(double value\_1, double value\_2, char token,
                     double *result);}

                     
                     {\color{red} \rule{\linewidth}{0.5mm}}


\begin{itemize}
\item @brief Функция совершает унарную операцию над значением

\item @param value значение, над которым будет производиться операция

\item @param token унарная операция (префиксная функция, унарный минус)

\item @param result значение, результат операции

\item @return код ошибки - (0 норм, 1 - не норм, выход за область определения функции)
\end{itemize}

\textcolor{blue}{int calculate\_unary(double value, char token, double *result);}


{\color{red} \rule{\linewidth}{0.5mm}}

\begin{itemize}
\item @brief Функция производит подсчет алгебраического выражения в постфиксной нотации.

\item @param expr строка, представляющая собой алгебраическое выражение, записанное в постфиксной нотации

\item @param result - переменная, куда будет записываться вычисленное значение

\item @return - код ошибки (0 - норм, 2 - деление на ноль, 1 - выход за границы области определения)
\end{itemize}

\textcolor{blue}{int calculate\_expression(char *expr, double *result);}


{\color{red} \rule{\linewidth}{0.5mm}}

\section{Introduction to frontend}

\subsection{\textbf{mainwindow.cpp}}

\begin{itemize}
\item @brief Конструктор класса MainWindow, наследуется от QWidget. Здесь происходит настройка событий, связанных с нажатием кнопок
\item @param parent Родитель нашего класса - QWidget
\end{itemize}

\textcolor{blue}{MainWindow::MainWindow(QWidget *parent);}

{\color{red} \rule{\linewidth}{0.5mm}}

\begin{itemize}

\item @brief Деструктор класса

\end{itemize}

\textcolor{blue}{MainWindow::~MainWindow();}

{\color{red} \rule{\linewidth}{0.5mm}}


\begin{itemize}

\item @brief Эта функция добавляет текст кнопки в выходную строку по некоторым правилам

\end{itemize}

\textcolor{blue}{void MainWindow::add\_str();}

{\color{red} \rule{\linewidth}{0.5mm}}


\begin{itemize}

\item @brief Функция очищает строку ввода

\end{itemize}

\textcolor{blue}{void MainWindow::clear\_line\_result();}

{\color{red} \rule{\linewidth}{0.5mm}}


\begin{itemize}

\item @brief Эта функция реагирует на изменение текста и убирает пробелы

\end{itemize}

\textcolor{blue}{void MainWindow::ignore\_whitespaces();}

{\color{red} \rule{\linewidth}{0.5mm}}


\begin{itemize}

\item @brief Эта функция удаляет один символ из строки ввода

\end{itemize}

\textcolor{blue}{void MainWindow::delete\_one\_char();}

{\color{red} \rule{\linewidth}{0.5mm}}


\begin{itemize}

\item @brief Эта функция производит подсчет алгебраического выражения в инфиксной нотации

\end{itemize}

\textcolor{blue}{void MainWindow::click\_equal\_sign();}

{\color{red} \rule{\linewidth}{0.5mm}}


\begin{itemize}

\item @brief Эта функция-валидатор. Проверяет первый введенный символ. Если символ не подчиняется определенным правилам, то он не вводится
\item @param str - строка ввода
\item @return - True/False

\end{itemize}

\textcolor{blue}{bool MainWindow::validate\_input\_empty\_line(QString str);}

{\color{red} \rule{\linewidth}{0.5mm}}


\begin{itemize}

\item @brief Это функция-валидатор, она не позволяет написать несколько операторов подряд
\item @param str - проверяемая строка
\item @return - True/False

\end{itemize}

\textcolor{blue}{bool MainWindow::validate\_double\_oper\_input(QString str);}

{\color{red} \rule{\linewidth}{0.5mm}}


\begin{itemize}

\item @brief Эта функция открывает диалоговое окно askrangewindow

\end{itemize}

\textcolor{blue}{void MainWindow::ask\_for\_range();}

{\color{red} \rule{\linewidth}{0.5mm}}


\begin{itemize}

\item @brief Эта функция вызывает диалоговое окно, изображается график функции из строки ввода и с заданными диапазонами
\item @param from\_X - левая граница области определения
\item @param to\_X - правая граница области определения
\item @param from\_Y - нижняя граница области значений
\item @param to\_Y - верхняя граница области значений

\end{itemize}

\textcolor{blue}{void MainWindow::mainGetPlot(double from\_X, double to\_X, double from\_Y, double to\_Y);}

{\color{red} \rule{\linewidth}{0.5mm}}


\begin{itemize}

\item @brief Эта функция вызывает диалоговое окно Credit\_calc

\end{itemize}

\textcolor{blue}{void MainWindow::getCredit();}

{\color{red} \rule{\linewidth}{0.5mm}}


\subsection{\textbf{askrangewindow.cpp}}


\begin{itemize}

\item @brief Конструктор класса с наследованием
\item @param parent 

\end{itemize}


\textcolor{blue}{AskRangeWindow::AskRangeWindow(QWidget *parent)}

{\color{red} \rule{\linewidth}{0.5mm}}

\begin{itemize}

\item @brief Деструктор класса

\end{itemize}

\textcolor{blue}{AskRangeWindow::~AskRangeWindow();}

{\color{red} \rule{\linewidth}{0.5mm}}

\begin{itemize}

\item @brief Функция вызывается после нажатия на кнопку btn\_ok
\item @param from\_X - левая граница области определения (ссылка)
\item @param to\_X - правая граница области определения (ссылка)
\item @param from\_Y - нижняя граница области значений (ссылка)
\item @param to\_Y - верхняя граница области значений (ссылка)

\end{itemize}

\textcolor{blue}{void AskRangeWindow::get\_range\_nums(
    double \&  from\_X, double \&  to\_X, double \& from\_Y, double \& to\_Y);}


{\color{red} \rule{\linewidth}{0.5mm}}

\begin{itemize}

\item @brief Закрытие окна

\end{itemize}

\textcolor{blue}{void AskRangeWindow::close\_win\_range();}

{\color{red} \rule{\linewidth}{0.5mm}}


\subsection{\textbf{credit\_calc.cpp}}

\begin{itemize}

\item @brief Конструктор класса Credit\_calc
\item @param parent родитель

\end{itemize}

\textcolor{blue}{Credit\_calc::Credit\_calc(QWidget *parent);}

{\color{red} \rule{\linewidth}{0.5mm}}


\begin{itemize}

\item @brief Деструктор класса

\end{itemize}

\textcolor{blue}{Credit\_calc::~Credit\_calc();}

{\color{red} \rule{\linewidth}{0.5mm}}


\begin{itemize}

\item @brief Отрисовка таблицы после ввода всех данных


\end{itemize}

\textcolor{blue}{void Credit\_calc::getTable();}

{\color{red} \rule{\linewidth}{0.5mm}}


\subsection{\textbf{my\_plot.cpp}}

\begin{itemize}

\item @brief Конструктор класса 
\item @param parent родитель
\item @param from\_X - левая граница области определения
\item @param to\_X - правая граница области определения
\item @param from\_Y - нижняя граница области значений
\item @param to\_Y - верхняя граница области значений
\item @param expression - функция

\end{itemize}


\textcolor{blue}{my\_plot::my\_plot(QWidget *parent, double from\_X, double to\_X, double from\_Y, double to\_Y, QString expression)}

{\color{red} \rule{\linewidth}{0.5mm}}


\begin{itemize}

                 
\item @brief Деструктор класса

\end{itemize}

\textcolor{blue}{my\_plot::~my\_plot();}

{\color{red} \rule{\linewidth}{0.5mm}}



\end{document}


